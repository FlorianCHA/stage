\documentclass[]{article}
\usepackage{lmodern}
\usepackage{amssymb,amsmath}
\usepackage{ifxetex,ifluatex}
\usepackage{fixltx2e} % provides \textsubscript
\ifnum 0\ifxetex 1\fi\ifluatex 1\fi=0 % if pdftex
  \usepackage[T1]{fontenc}
  \usepackage[utf8]{inputenc}
\else % if luatex or xelatex
  \ifxetex
    \usepackage{mathspec}
  \else
    \usepackage{fontspec}
  \fi
  \defaultfontfeatures{Ligatures=TeX,Scale=MatchLowercase}
\fi
% use upquote if available, for straight quotes in verbatim environments
\IfFileExists{upquote.sty}{\usepackage{upquote}}{}
% use microtype if available
\IfFileExists{microtype.sty}{%
\usepackage{microtype}
\UseMicrotypeSet[protrusion]{basicmath} % disable protrusion for tt fonts
}{}
\usepackage[margin=1in]{geometry}
\usepackage{hyperref}
\hypersetup{unicode=true,
            pdftitle={Rapport de Stage},
            pdfauthor={CHARRIAT Florian},
            pdfborder={0 0 0},
            breaklinks=true}
\urlstyle{same}  % don't use monospace font for urls
\usepackage{graphicx,grffile}
\makeatletter
\def\maxwidth{\ifdim\Gin@nat@width>\linewidth\linewidth\else\Gin@nat@width\fi}
\def\maxheight{\ifdim\Gin@nat@height>\textheight\textheight\else\Gin@nat@height\fi}
\makeatother
% Scale images if necessary, so that they will not overflow the page
% margins by default, and it is still possible to overwrite the defaults
% using explicit options in \includegraphics[width, height, ...]{}
\setkeys{Gin}{width=\maxwidth,height=\maxheight,keepaspectratio}
\IfFileExists{parskip.sty}{%
\usepackage{parskip}
}{% else
\setlength{\parindent}{0pt}
\setlength{\parskip}{6pt plus 2pt minus 1pt}
}
\setlength{\emergencystretch}{3em}  % prevent overfull lines
\providecommand{\tightlist}{%
  \setlength{\itemsep}{0pt}\setlength{\parskip}{0pt}}
\setcounter{secnumdepth}{0}
% Redefines (sub)paragraphs to behave more like sections
\ifx\paragraph\undefined\else
\let\oldparagraph\paragraph
\renewcommand{\paragraph}[1]{\oldparagraph{#1}\mbox{}}
\fi
\ifx\subparagraph\undefined\else
\let\oldsubparagraph\subparagraph
\renewcommand{\subparagraph}[1]{\oldsubparagraph{#1}\mbox{}}
\fi

%%% Use protect on footnotes to avoid problems with footnotes in titles
\let\rmarkdownfootnote\footnote%
\def\footnote{\protect\rmarkdownfootnote}

%%% Change title format to be more compact
\usepackage{titling}

% Create subtitle command for use in maketitle
\newcommand{\subtitle}[1]{
  \posttitle{
    \begin{center}\large#1\end{center}
    }
}

\setlength{\droptitle}{-2em}

  \title{Rapport de Stage}
    \pretitle{\vspace{\droptitle}\centering\huge}
  \posttitle{\par}
    \author{CHARRIAT Florian}
    \preauthor{\centering\large\emph}
  \postauthor{\par}
      \predate{\centering\large\emph}
  \postdate{\par}
    \date{29 mars 2018}


\begin{document}
\maketitle

\section{Matériels et Methodes}\label{materiels-et-methodes}

\subsection{1.Matériels}\label{materiels}

\begin{quote}
\begin{itemize}
\tightlist
\item
  201 génome (81 litterature et 120 projet)
\item
  Définir les 32 RNAseq
\item
  Définir protéine de référence de 70-15 (expliquer 70-15)
\end{itemize}
\end{quote}

\subsection{2.Methodes}\label{methodes}

\subsubsection{2.1 Assemblage}\label{assemblage}

Les données de séquençage étant des données short read, l'outils ABySS
(Assembly By Short Sequences) {[}4{]} à été utilisé pour réaliser les
assemblages. L'assemblage avec cet outil se réalise en deux étapes. Une
première étape consiste à generer touts les k-mère de longeur donnée à
partir des reads. Cette ensemble de k-mères est ensuite utilisée pour
construire les contigs initiaux. Puis la deuxième étape utilise les
informations de pair end pour étendre les contigs et résoudre les
ambiguités de chevauchement des contigs.

Pour obtenir le meilleur assemblage possible, chaque souche est
assemblée 8 fois avec une taille de kmère différents. Les tailles de
kmère utilisées vont de 20 a 90 avec un pas de 10. Le meilleur
assemblage de chaque souche est ensuite sélectionné après visualisation
des données de qualité.

Une fois la selection réalisé, le script formatFastaName est utilsé.
Pour chaque assemblage selectionné, le script élimine, dans une premier
temps, les scaffolds de taille inférieur à 500 pb qui risque de poser
problème pour l'annotation. Puis les scaffolds seront numerotés en
fonction de leur longueur afin d'homogeneisé la nomenclature des
scaffolds. Le plus grand scaffold sera alors nommé scaffold\_1.

\subsubsection{2.2 Eléments répétés}\label{elements-repetes}

Une fois l'assemblage réalisé, il est necessaire de masquer les éléments
répétés. En effet, les éléments répétés peuvent poser problèmes lors
d'annotation automatique. Ils peuvent être confondus avec des gènes
codant pour des protéines. Mais peuvent aussi perturber la structure des
modèles de gène, en s'insérant dans les introns par exemple. L'outil
repeatMasker est donc utilisé pour masquer les éléments répétés des
assemblages selectionnés. Cet outil compare les sequences provenant
d'une base de données d'éléments répétés avec ceux provenant de
l'assemblage. Il va ensuite masquer ces éléments trouvés en les
remplaçant par des N. L'outils repeatMasker permet aussi d'obtenir une
annotation détaillée des répétitions présentes dans l'assemblage. La
base de donnée utilisé pour repeatMasker est Repbase auquel à été ajouté
les éléments répétées decouvert par l'equipe à l'aide de l'outil
RepeatModeler.

Une fois les éléments masqués masqué, les scaffolds ne comportant que
des éléments répétés sont élimner des assemblages. En effet, les outils
utilisés pour l'annotation automatique n'arrive pas a annoter les
assemblages comportant des scaffolds uniquement composé de N.

\subsubsection{2.3 Annotation automatique}\label{annotation-automatique}

M.oyzae étant un eukaryote, l'outil Braker à été utilisée pour
l'annotation automatique. Cet outil nescessite un fichier d'assemblage
au format fasta et un fichier d'alignement de donnée RNA-seq sur
l'assemblage au format bam. Braker utilise un pipeline d'annotation qui
utilise les outils GeneMark-ET et AUGUSTUS.

Tout d'abord, GeneMark-ET génère des structures géniques. Pour cela
l'outil utilse un ensemble de paramètre du hidden semi-Markov model
(HSMM) définit initiallement pour prédire des régions codante dans
l'assemblage. Une fois ces gènes prédites, un sous ensemble d'exon et
intron est utilisé pour re-estimer les paramètres du HSMM. L'intégration
des région codante dans le set d'entrainement nécessite que l'exon
prédit possède au moins un site d'épissage. Les sites d'épissage sont
ceux prédits indépendamment par les deux méthodes, l'ab initio et
l'alignement des données RNA-Seq. Les étapes de prédiction des régions
condantes et de la ré-estimation des paramètres du HSMM sont réalisées
de manière itérative tant que les regions condantes prédites varies.

Deuxièmement, AUGUSTUS utilise les gènes prédits par GeneMarker-ET pour
l'entraînement, puis intègre les informations de mapping des données
RNA-seq dans les prédictions de gènes finaux.

Pour enrichir les données utilisées par Braker le fichier d'alignement
bam est remplacé par un fichier hints intronique. Ce fichier correspond
au position de tous les introns récupérés à partir de fichiers
d'alignements. Le fichier hints utilisés est issue de l'alignement des
32 RNA-seq et du fichier de protéine de références issue de l'annotation
de 70-15.

Pour l'annotation automatique des assemblages, le pipeline
BRAKER\_pipeline à été réalisé en snakemake. La figure X présente un
exemple du pipeline d'annotation pour la souche XXX. Les alignements des
données RNA-seq et l'alignement des protéines de référence vont être
traité en parrallèle par le pipeline. Pour l'alignement des 32 données
de RNA-seq Braker\_pipeline va être utilisé l'outil topHat2. Puis les
fichiers bam obtenue seront mergé à l'aide de l'outils samtools merge.
En suite le fichier contenant tous les alignements va être trié à l'aide
de l'outils picard SortSam. Enfin l'alignement au format bam va être
converti en fichier hints à l'aide de l'outil bam2hints fournis avec
l'outil Braker. Pour l'alignement des protéines de référence, l'outil
exonerate va être utilisé. Une fois l'alignement réalisé l'outil
exonerate2hints est utilisé pour convertir le fichier d'alignements en
fichier hints.


\end{document}
